\documentclass{beamer}

\usetheme{default}

\usepackage{tcolorbox}

\usepackage{hyperref}
\hypersetup{
    colorlinks=true,
    linkcolor=blue,
    filecolor=magenta,      
    urlcolor=cyan
}

\title{Machine Readable Repository}
\subtitle{Lessons from CORD-19}
\author{Marcelo Garcia}
\institute{KAUST University Library}
\date{\today}

\begin{document}

\begin{frame}
\titlepage
\end{frame}

\begin{frame}{What is CORD-19?}
    \begin{itemize}
        \item CORD-19: COVID-19 Open Research Dataset
        \item Large collection of publications and preprints on COVID-19, and related virus like SARS and MERS
        \item The goal was connect the machine learning community with biomedical experts and policy makers    
    \end{itemize}
\end{frame}

\begin{frame}{Who is involved?}
    \begin{itemize}
        \item Allen Institute for AI
        \item The White House Office of Science and Technology Policy
        \item National Library of Medicine
        \item Chan Zuckerberg Initiative
        \item Microsoft Research
        \item Kaggle
        \item Georgetown University's Center for Security and Emerging Technology (Coordination)
    \end{itemize}
\end{frame}

\begin{frame}{How does it work?}
    \begin{itemize}
        \item Metadata and documents (depending on license) ingested with Semantic Scholar
        \item Pipeline: $PDF \rightarrow XML \rightarrow JSON$ \footnote{Kyle Lo, Lucy LuWang, Mark Neumann, Rodney Kinney, and Daniel S.Weld. 2020. \href{https://arxiv.org/abs/1911.02782}{S2ORC: The Semantic Scholar Open Research Corpus}. In Proceedings of ACL.}
    \end{itemize}
    \begin{tcolorbox}[colback=red!5!white,colframe=red!75!black,title=Challenges]
    \small
    \begin{itemize}
        %\item PDF format not suitable for NLP (Natural Language Processing)
        \item Significant effort is needed to coerce PDF into a format more amenable to text mining, like S2ORC JSON.
        \item Not standard format to represent metadata. Formats like \texttt{BIBFRAME} or Dublin Core can be too coarse grained.
    \end{itemize}
    \end{tcolorbox}
    
\end{frame}

\begin{frame}{Why make the repository machine readable?}
    \begin{description}
        \item [Tim Berners-Lee's vision] for the next Web: machines talking with machines (``The next Web of open, linked data'' - TED talk)
        \item [Information extraction] extract key information (entity) in the text, like a specific name or coral species
        \item [Text classification] extract sentences or passage of interest from the text
        \item [Knowledge graphs] for a specific domain, like \href{https://healthecco.org/covidgraph/}{HealthEcco}
    \end{description}

    \begin{tcolorbox}
        We are providing only the datasets, not doing any AI/ML thing.
    \end{tcolorbox}
\end{frame}

\end{document}
